% What's the problem?
% Why is it a problem? Research gap left by other approaches?
% Why is it important? Why care?
% What's the approach? How to solve the problem?
% What's the findings? How was it evaluated, what are the results, limitations, 
% what remains to be done?

\emph{Summary:}
We'll introduce literate programming: what it is, one tool and a workflow to 
use it.
We'll also discuss the potential benefits of literate programming, and how it 
relates to different techniques used in programming.

\emph{Intended learning outcomes:}
We intend that the student reach the following learning objectives.
\ltnote{%
  In terms of variation theory \parencite{NecessaryConditionsOfLearning}, we 
  can divide these learning objectives into several aspects.
  The following is a first approximation of what those aspects are.
  We'll need to refine this as we learn how students perceive this.
  LO~\ref{LOlitprog} covers:
  \begin{enumerate}
    \item Particularly distinguishing the \enquote{webbiness} of literate 
      programming and compare this to docstrings, notebooks and other similar 
      technologies.
    \item Writing the code for a human, rather than how they normally write 
      code.
      That is, using the full power of \web-based languages for the purpose of 
      exposition.
      This includes how to use literate programming to bring related things 
      together, for example tests and the code it tests.
  \end{enumerate}
}%
\begin{restatable}{lo}{LOlitprog}\label{LOlitprog}
  The student can distinguish the core features of literate programming and how 
  they compare to similar technologies.
\end{restatable}
\ltnote{%
  LO~\ref{LOnoweb} covers several aspects:
  \begin{enumerate}
    \item Tangle the literate program into the machine-readable code.
      Which in turn concerns several aspects; such as how to have several files 
      in one file, use the \noweb{} langugage, use tools such as \texttt{noerr} 
      to translate error locations.
    \item Weave the literate program into a human-readable document.
  \end{enumerate}
}%
\begin{restatable}{lo}{LOnoweb}\label{LOnoweb}
  The student can use \texttt{noweb} to write literate programs.
\end{restatable}

\emph{Prerequisites:}
The student should be able to write programs using a language like Python and 
be able to run them.
The student should also be able to write documents using \LaTeX.
Finally, the student should be able to use a terminal.

\emph{Reading:}
The material is self-contained, but the student may find it useful to read the 
references at the end.


\section{Related approaches}\label{RelatedApproaches}

Let's start by thinking of systems that we know that might fulfil some of these 
goals.%
\ltnote{%
  As mentioned above, we want to show how well different methods do and don't 
  fulfil the goals of literate programming.
  This is to get a better understanding for the goals themselves, rather than 
  the different methods.

  From a variation theoretic perspective, the students must see examples that 
  both fulfils and don't fulfil the goals.
}%
\ltnote{%
  The purpose of \cref{NameOtherRelated} is to get them to think about systems 
  they already know that might come close to this.
  From a variation theoretic perspective, they now think about why or why not 
  some things fulfil these goals.
  This means that they have already gotten started on the contrast pattern.
  Compare with the Japanese classrooms discussed in Chapter 6 of 
  \cite{NecessaryConditionsOfLearning}.

  However, we don't ask them to write anything at this point.
  To name them, they still have to think about this.
  Then we'll go through a few in detail.
  Then we can ask them for remaining ones.
}%
\begin{frame}
\only<presentation>{\LPgoals}
\begin{activity}\label{NameOtherRelated}
Do you know of any tools/systems/languages that lets you do something like 
this?
Name them.
\end{activity}
\end{frame}

We'll now cover a few such systems:
\ltnote{%
  This provides a reasonable overview of some of the possible answers.
  If they answered and know one of the them, they do get enough feedback.
  If they don't, we'll go into the details of them in the following text.
}%
\begin{enumerate*}[label=(\arabic*)]
\item notebooks,
\item docstrings,
\item rubber-ducking or rubber-duck debugging,
\item pair programming,
\item documentation in version control systems, and
\item pseudocode.
\end{enumerate*}
There might be more than those that I cover here, but these are among the most 
well-known ones.

\subsection{Notebooks}

Probably the first that comes to mind is the Jupyter notebook.
As you can see in \cref{fig:jupyter}, a notebook consists of text interspersed 
by code that can be executed (usually to generate graphics).

\begin{frame}<presentation>
\begin{activity}\label{AnalyseNotebooks}
How well do you know the concept of notebooks?
(Be it Jupyter, Google Colab, or any other similar system.)
\end{activity}
\end{frame}

In more detail, notebooks consists of cells, which can be of different types.
Some might be text and others are code.
The code cells can be executed and the output is shown below the cell, in the 
document (this is what we see in \cref{fig:jupyter}).
This way the author can explain the code in the text cells and then show the 
code in the code cells.

The code cells can be executed in any order, which is both a strength and a 
weakness.
It's a strength because you can experiment and try things out.
It's a weakness because you can't really successively build the program.
In some configurations, however, they can build on each other.
However, in this case, we get dependencies and the cells must be executed in 
the right order (if not chronologically, we get \enquote{spaghetti code} with 
global variables).

\begin{frame}
\begin{figure}[b]
  \includegraphics[width=\columnwidth]{figs/jupyter-notebook.png}
  \caption{\label{fig:jupyter}%
    The Jupyter notebook.
  }
\end{figure}
\end{frame}

\begin{frame}
\begin{remark}
  The concept of notebooks seems to originate with 
  Mathematica\autocite{Wolfram1988}.
  Wolfram started the work in 1986, two years after Knuth published 
  \citetitle{Knuth1984}.
  \only<article>{%
    It's unclear from the material that I've reviewed so far, if Wolfram was 
    inspired by Knuth's work.
    It might be independent due Wolfram being a physicist and Mathematica's 
    focus on mathematics.
    The notebook concept is basically just an interactive (and computational) 
    extension of the traditional way of using mathematics in physics (papers 
    and notebooks).%
  }
\end{remark}
\end{frame}

\begin{frame}
\ltnote{%
  \Cref{NotebooksFullfilGoals} allows the students to compare notebooks to the 
  goals of literate programming.
  This is the start of a generalization pattern:
  we keep the goals invariant and vary the systems/tool/approach.
  (We did get the contrast above, in \cref{WhatDoesLPMean}.)

  However, to do this, we'd better explain briefly what the technology or 
  system actually is.
  The reader might not know, or have only limited knowledge of it.
}%
\only<presentation>{\LPgoals}
\begin{activity}\label{NotebooksFullfilGoals}
How well do you think notebooks fulfil the goals of literate programming?
\end{activity}
\end{frame}

It for sure fulfils the first goal, \enquote{\LPexplain}.
However, it doesn't necessarily fulfil the second goal, \enquote{\LPorder}.
We can't really have any order we want, we must follow the order of execution 
in the cells---which is ordered for the computer, not the human.
(Although, these might align at times, but at other times they don't.)

So in summary, we have:

\begin{frame}
\begin{block}{Notebooks}
  \begin{itemize}
    \item[\(+\)] {\color{green!75!black}\LPexplain}
    \item[\(-\)] {\color{red}\LPorder}
  \end{itemize}
\end{block}

\begin{remark}
  Code is in order of the computer.
\end{remark}
\end{frame}

\subsection{Docstrings}

Docstrings is another concept that has been argued is literate programming.
The reasoning is that we generate documentation from the code.
Consider the documentation for the function [[sorted]]\label{sortedDocExample} 
in Python:
\begin{frame}[fragile]
\begin{pycode}
didactic_pydoc("sorted", builtin=True)
\end{pycode}
\end{frame}
The documentation here is generated from the function's code:
\begin{frame}[fragile]
\begin{minted}{python}
def sorted(iterable, /, *, key=None, reverse=False):
    """
    Return a new list containing all items from the iterable in ascending 
    order.

    A custom key function can be supplied to customize the sort order, and the
    reverse flag can be set to request the result in descending order.
    """
    ...
\end{minted}
\end{frame}

\begin{frame}
\only<presentation>{\LPgoals}
\begin{activity}
How well do you think docstrings fulfil the goals of literate programming?
\end{activity}
\end{frame}

The docstrings aim at explaining how to use a particular function or class.
It explains what the inputs and outputs are, but not how those are achieved.
The purpose of them is to describe how to use these constructs as black boxes, 
without having to know how they do it.

Nothing in the documentation explains how the sorting is done, no details of 
the underlying sorting algorithm(s).

We could of course generate an explanation for a piece of code by extracting 
the docstrings of the functions being called.
However, that only explains what the code does, same as reading the code 
itself.
It doesn't explain the programmer's intent when writing the code.

In summary:

\begin{frame}
\begin{block}{Docstrings}
  \begin{itemize}
    \item[\(-\)] {\color{red}\LPexplain}
    \item[\(-\)] {\color{red}\LPorder}
  \end{itemize}
\end{block}

\begin{remark}
  Explains what, not why.
\end{remark}
\end{frame}

\subsection{Rubber duck debugging}

Another related concept is rubber duck debugging.

\begin{frame}<presentation>
\begin{activity}
How well do you know rubber duck debugging?
\end{activity}
\end{frame}

The idea of rubber duck debugging is that you explain the code for a rubber 
duck (or any other object or person that can't help you solve the problem).
Just by explaining, you get a better understanding of the code, which helps you 
to find the bug more easily.

\begin{frame}<presentation>
\begin{block}{Rubber duck debugging}
  \begin{itemize}
    \item Explain the code to a rubber duck.
    \item Helps understanding the code to find the bug.
  \end{itemize}
\end{block}
\end{frame}

\begin{frame}
\only<presentation>{\LPgoals}
\begin{activity}
How well do you think rubber duck debugging fulfils the goals of literate 
programming?
\end{activity}
\end{frame}

This for sure comes close to the goals of literate programming:
We explain the details of the code, why we've done the way we did and what we 
expected.
We can also explain the code in the order we want, to fit the human 
understanding rather than the computer's execution.

\begin{frame}<presentation>
\begin{block}{Rubber duck debugging}
  \begin{itemize}
    \item[\(+\)] {\color{green!75!black}\LPexplain}
    \item[\(+\)] {\color{green!75!black}\LPorder}
  \end{itemize}
\end{block}
\end{frame}

However, none of this is documented, so it's lost to others (and our future 
selves).
And we do spend more time reading code than writing it, so this is quite a 
loss: \blockcquote[p.~14]{CleanCode}{%
  Indeed, the ratio of time spent reading vs.~writing is well over 10:1.
  We are constantly reading old code as part of the effort to write new code.%
}
Also, the programmer is not encouraged to explain the details unless there is a 
bug to find.

\begin{frame}
\begin{block}{Rubber duck debugging}
  \begin{itemize}
    \item[\(-\)] {\color{red}\LPexplain}
    \item[\(-\)] {\color{red}\LPorder}
  \end{itemize}
\end{block}
\begin{remark}
  None is documented, so it's lost to others.
\end{remark}
\end{frame}

\subsection{Pair programming}

Pair programming is similar to rubber duck debugging, but with another human 
instead of a rubber duck and you do this while developing---not just when 
debugging.
So that's an improvement.

\begin{frame}<presentation>
\begin{block}{Pair programming}
  Pair programming is sort of the same, but with a fellow human instead of a 
  rubber duck.
\end{block}

\begin{block}{Pair programming}
  \begin{itemize}
    \item[\(+\)] {\color{green!75!black}\LPexplain}
    \item[\(\approx\)] {\color{orange}\LPorder}
  \end{itemize}
\end{block}
\end{frame}

The two programmers working together explain the thoughts of the design and 
code to each other as they progress the development.
(To be exact there is a \enquote{navigator}, who explains to the 
\enquote{driver}, who writes the code.)

To some extent, they can do this in the order needed.
However, in the end, the code must be written in the order the computer 
understands it.
And in the end, nothing from this process is necessarily stored as 
documentation.
So it's again lost to others, including the programmers' future selves.

\begin{frame}
\begin{block}{Pair programming}
  \begin{itemize}
    \item[\(-\)] {\color{red}\LPexplain}
    \item[\(-\)] {\color{red}\LPorder}
  \end{itemize}
\end{block}

\begin{remark}
  They have to write the code in the order the computer understands it.
  And it's not necessarily documented, so it's lost to others.
\end{remark}
\end{frame}

\subsection{Documentation in version management}

We can also document meaning in the version management system, as commit 
messages and notes in issues and pull requests.

\begin{frame}
\only<presentation>{\LPgoals}
\begin{activity}
How well do you think documentation in version management systems fulfil the 
goals of literate programming?
\end{activity}
\end{frame}

The documentation of \emph{changes} in the version management system sure 
contains good explanations of why the changes are needed.
For instance, [[git blame]] and similar commands, allows us to see a motivation 
for each line of code.

However, this does not necessarily give us the order needed to understand the 
program.
But rather as the commit diff was generated, together with one block of 
explanation.
So the quality depends on the sizes of the commits.

Another problem is that the commits for the first version don't necessarily 
contain the same quality of explanation as the later commits.
For those we are probably left with only the comments in the code.
And in this case, the order of exposition is adapted for the computer rather 
than humans.

\begin{frame}
\begin{block}{Documentation in version management}
  \begin{itemize}
    \item[\(\approx\)] {\color{orange}\LPexplain}
    \item[\(\approx\)] {\color{orange}\LPorder}
  \end{itemize}
\end{block}

\begin{remark}
  Documents changes mostly, not first version.
  Exposition in diffs.
\end{remark}
\end{frame}

\subsection{Pseudocode}

The last one that we'll cover is pseudocode.

\begin{frame}
\only<presentation>{\LPgoals}
\begin{activity}
How well do you think pseudocode fulfils the goals of literate programming?
\end{activity}
\end{frame}

Pseudocode actually comes quite close.
It's a way to explain the code to humans, and it's not necessarily tied to the 
order of execution.
We can vary the level of detail and the order of exposition to fit the human 
understanding.
As something usually used in the design process, it helps with the 
understanding of the overall design.
However, it's not necessarily integrated with the code\footnote{%
  Unless we use the methodology proposed by \textcite[Ch.~9]{CodeComplete}.
  However, even then, we don't keep the order of exposition adapted to humans.
  When we use the pseudocode as the comments of the code, the order is dictated 
  by the computer's understanding.
}.
And in the worst case, it's thrown away once there is an implementation.

\begin{frame}<presentation>
\begin{block}{Pseudocode}
  \begin{itemize}
    \item[\(+\)] {\color{green!75!black}\LPexplain}
    \item[\(+\)] {\color{green!75!black}\LPorder}
  \end{itemize}
\end{block}
\end{frame}

\begin{frame}
\begin{block}{Pseudocode}
  \begin{itemize}
    \item[\(\approx\)] {\color{orange}\LPexplain}
    \item[\(\approx\)] {\color{orange}\LPorder}
  \end{itemize}
\end{block}

\begin{remark}
  But this is not necessarily kept in the documentation.
  It's also not necessarily integrated with the code.
\end{remark}
\end{frame}

%\subsection{Testing}

\subsection{Others}

This covered only some of the related concepts.

\ltnote{%
  We asked them in the beginning to name some systems that they think are 
  related.
  Some of their answers were probably covered by the examples above.
  In either case, they did their own analysis above, an analysis that might 
  have been complemented by the examples we gave.
  Now we give them the chance to provide us with the complemented analyses from 
  before, for the systems that we didn't cover.
}%
\begin{frame}
\only<presentation>{\LPgoals}
\begin{activity}
Do you know of any other tools/systems/languages/techniques that let you do 
something like this?
Name them and discuss how well they fulfil the goals.
\end{activity}
\end{frame}




Now, we'll get the same when debugging using \texttt{gdb}, it will refer to the 
literate source instead of the tangled file.
(I've hightlighted some lines to make it easier to read the following output.)
\begin{minted}[highlightlines={18,41,43,48}]{text}
$ gdb ./fracexample
GNU gdb (Ubuntu 12.1-0ubuntu1~22.04.2) 12.1
Copyright (C) 2022 Free Software Foundation, Inc.
License GPLv3+: GNU GPL version 3 or later <http://gnu.org/licenses/gpl.html>
This is free software: you are free to change and redistribute it.
There is NO WARRANTY, to the extent permitted by law.
Type "show copying" and "show warranty" for details.
This GDB was configured as "x86_64-linux-gnu".
Type "show configuration" for configuration details.
For bug reporting instructions, please see:
<https://www.gnu.org/software/gdb/bugs/>.
Find the GDB manual and other documentation resources online at:
    <http://www.gnu.org/software/gdb/documentation/>.

For help, type "help".
Type "apropos word" to search for commands related to "word"...
Reading symbols from ./fracexample...
(gdb) l 35
warning: Source file is more recent than executable.
30      @
31
32      \subsection{C++}
33
34      The class should be used as the following example programs show.
35      <<fracexample.cpp>>=
36      #include <iostream>
37      #include "fraction.h"
38
39      int main() {
(gdb) l
40        Fraction f1(1, 2);
41        Fraction f2(1, 3);
42        std::cout << f1 << " + " << f2
43                  << " = " << (f1 + f2) << std::endl;
44      }
45      @
46
47      The expected output is:
48      \begin{pycode}
49      didactic_shell("./fracexample", minted_opts="numbers=none")
(gdb) b 42
Breakpoint 1 at 0x2411: file /home/dbosk/devel/edu/literate-programming/whatis/cppjava.nw, line 42.
(gdb) r
Starting program: /home/dbosk/devel/edu/literate-programming/whatis/fracexample 
[Thread debugging using libthread_db enabled]
Using host libthread_db library "/lib/x86_64-linux-gnu/libthread_db.so.1".

Breakpoint 1, main () at /home/dbosk/devel/edu/literate-programming/whatis/cppjava.w:42
42        std::cout << f1 << " + " << f2
(gdb) 
\end{minted}
We see above that the breakpoint refers to lines in the literate source.
And, indeed, when we ask \texttt{gdb} to list the source around the breakpoint, 
it lists the literate source.
(We used \texttt{gdb} in the example, this is what IDEs like VSCode do as well, 
so it will work there too.)
